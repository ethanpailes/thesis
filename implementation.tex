\chapter{An Implementation}
\label{chapter:implementation}

This chapter details our implementation of a regular expression
engine which leverages skip regex. It pays particular attention
to the implementation of the abstract VM outlined in section
\ref{section:vm}. We implemented skip regex as
an extension to Rust's Regex crate. The decision to extend rather
than start a fresh project allowed us to reuse the parser,
testing tools, and even a few implementation techniques. Extending
an existing library made it even easier to provide backwards
compatibility with an existing ecosystem. Our implementation
can be found on the web at \cite{PailesSkipRegex}.

\section{Architecture}

Executing a skip regular expression requires a number of phases
arranged in a pipeline. Before any work can be done on the regex,
it must be parsed. Fortunately, the backwards compatible concrete
syntax of skip regex means that we could just use the parser provided
by Rust's regex crate. The real work of executing skip regex does not
pick up until it is time to compile the regular expression.

Skip regex are compiled by applying the \verb'code' function defined
in the section \ref{section:compilation}. In practice, we make a
few refinements to the compilation technique for performance.
The biggest such change
is to the format of the VM instructions that we execute. Rather than
using explicit labels, each instruction is annotated with one or
two \verb'goto' pointers indicating the next instruction to jump to.
This enables the complete removal of the \verb'jmp' instruction, which
cuts down on both code size and the number of VM iterations required to execute
a particular regex. This new format requires continually back-patching
the last instruction or instructions during compilation, which is a bit
less clear than the direct style used to define the \verb'code' function.

Once a regular expression is compiled, it can be executed by one
of many potential engines, each with its own set of trade offs.
The main engine we use is a bounded backtracking engine. The
implementation is covered in greater detail below, but the core
idea of a bounded backtracker is to extend a standard backtracker
with a bitset indicating which states the engine has already
been in. So long as history does not matter (which is the case
so long as back-references are not allowed), it is safe to skip any
state which has already been visited. Doing so has the benefit
of restoring guaranteed linear running time.

The bitset that allows the bounded backtracker to provide guaranteed
linear running time is $O(|input||regex|)$ in space, which is
cheap when the regex and input are small but can get quite expensive
as they grow. Rob Pike's lockstep VM does not suffer from this
space blowup, so it makes a good fallback despite running slower
for small cases. Unfortunately, Pike's VM cannot handle branch
precedence correctly when extended with the ability to execute
skip instructions. Match disambiguation is not always required,
such as in the case of the expression
\verb'/remind me (tonight|tomorrow)/',
so Pike's VM can still sometimes be useful when extended with
skip instructions (\cite{Pike1987}).

\section{Backwards Incompatibilities or Missing Features}

The Rust regex library is a mature and full featured regular expression
engine, which provided significant opportunity to reuse well tested
components, but it also meant that bringing a new regex engine to
full feature parity with the existing pipeline was difficult. We
choose to defer implementation of a few features that the regex crate
supports in order to focus on exploring the key concepts in play.
We discuss these features briefly here to provide a sketch of how
they might be added to skip regex.

The first feature we omit is support for case-insensitive regex.
Many regex engines provide a flag which asks the engine to ignore
case when performing a match. Case-insensitivity is useful when searching
for words embedded in natural language or keywords of a case-insensitive
language like SQL. Implementing this feature can be performed as
an AST rewrite pass, where each character is replaced by a character
class which includes both the character and its capital version.
For example \verb'/abc/' would be rewritten as \verb'/[aA][bB][cC]/'
\footnote{In more recent versions of the regex crate than the one
that we forked, we would have gotten this feature for free.}.
We opted not to implement this feature to save time,
but there is no reason it should not work with skip regex.

The second feature we omit is unicode support. Unicode support in
a regex library is non-trivial, primarily due to the fact that
unicode code-points are variable width, which makes indexing into
a unicode string $O(n)$ in the size of the string. The scan optimizations
on skip regex would be unaffected by loosing access to constant time
indexing, because they have to deal
with scanning forward over chunks of the input anyway. Even in the
context of a unicode string, substring search algorithms can operate at
the byte level. The implementation of the skip instruction would require
a bit more work. The simplest thing to do would be to just construct
a unicode DFA\footnote{Or more likely, just reuse an existing one.}
to decode the input, and then use it to scan quickly
forward, which would take the runtime of the skip instruction from
$O(1)$ to $O(n)$, but the constant factor would still be lower
than the cost associated with an NFA simulation. We could improve on
this situation significantly by adding a separate \verb'skip-exact'
instruction for use in situations where the size in bytes of the
skip can be statically determined. Thus, a literal unicode character
which takes up three bytes would compile down to a \verb'skip-exact 3'
instruction. Adding support for unicode to skip regex would be significantly
more work than adding support for case insensitivity, but would still be
perfectly feasible.

\section{Skip Bounded Backtracker}

The primary backend used to execute skip regex is the skip bounded
backtracker. Backtracking regex engines have a long history.
Their speed, ease of implementation, and extensibility make them
a great choice for a regex backend, but they suffer from a worst-case
exponential running time. The exponential blow up can be fixed by
keeping a bitset which tracks every state that the engine has
ever been in, where a state is identified by the tuple of the
instruction pointer and the input index. Once the engine has
been in a particular state once, there is no need to explore it
again.

The primary cost associated with the bounded backtracker is
the initialization of the bitset, which consumes $O(|regex||input|)$
space. Very few executions cause the NFA simulation to enter all
or most states, but initializing the bitset to all zeros must always
touch every state bit. When the expression and input are small, the
cost is negligible, but as they grow it eventually starts to dominate.
One subtle benefit of the skip regex optimizations we perform is that
they all have the effect of reducing code size,
which cuts down on the size of the bitset, and hence the cost of
its initialization.

\subsection{Algorithm}

The bounded backtracking algorithm can be very naturally expressed
in terms of a recursive function parameterized by a set of captures,
an instruction pointer, and a string offset.  The ``Backtrack'' procedure
laid out in algorithm \ref{algo:recurboundedback} takes an array of capture
slots, an instruction pointer and a string pointer as input and returns
either ``NoMatch'' to indicate that no match could be found
or ``Match($captures$)'' to indicate that there was a match with the
given capture slots. Some pre-processing is required to
set up the bitset (here called $seen$) and the capture slots,
and some post processing is required to convert the capture slots
to capture groups suitable for user consumption, but the core
of the algorithm is here. Strictly speaking, ``Backtrack'' should
never return ``NoMatch'' because it is invalid to invoke a skip
regex on input which will not match the regex, but the concept of
a failed intermediary match is required to support non deterministic
branching internally (the \verb'split' instruction). There are also
some cases where skip regex \emph{can} detect invalid input at no
extra cost, and there is no particular reason to avoid telling the
user about such cases.

Algorithm \ref{algo:recurboundedback} is presented at a
high level, and there are a number of tricks that we use in the
real implementation which improve performance at the cost of
readability. We use an explicit bitset rather than an abstract
set to implement the seen-set. Rather than using explicit recursion,
we use a heap-allocated stack of actions representing stack frames
and a direct loop.
Crucially, we avoid copying the $captures$ array at every step by
modifying a single captures array in place and introducing a restoration
action which can be interleaved with the standard stack resumption action.
Similarly, the instruction pointer and string pointer are modified
in place. All of these implementation tricks were inspired by the
bounded backtracker already found in the rust crate (\cite{GallantRegex}).
Algorithm \ref{algo:loopboundedback} implements these
optimizations. Several of the less interesting instructions are
omitted for brevity, but it should be easy to see how the implementation
of algorithm \ref{algo:loopboundedback}
might be translated from algorithm \ref{algo:recurboundedback} in
such cases.

\begin{algorithm}
\caption{Recursive Bounded Backtracker}
\label{algo:recurboundedback}

\begin{algorithmic}
\State Assume the existence of a $scan\_to\_start(needle, haystack)$ function
        which performs fast substring search for the needle in the haystack
        (using Tuned Boyer-Moore (\cite{Hume1991}),
        or hardware accelerated literal search as appropriate).
\State Let $input$ be the input string.
\State Let $insts$ be the instructions.
\State $seen \gets \{\}$
\Procedure{Backtrack}{$captures$, $ip$, $sp$}
  \If{$\langle ip, sp \rangle \in seen$}
    \State \textbf{return} NoMatch
  \EndIf
  \State $seen \gets seen \cup \{\langle ip, sp \rangle\}$
  \Switch{$insts[ip]$}
    \Case{{\tt split L1 L2}}
      \Switch{Backtrack($captures$, {\tt L1}, $sp$)}
        \Case{NoMatch}
          \State \textbf{return} Backtrack($captures$, {\tt L2}, $sp$)
        \EndCase
        \Case{Match($captures'$)}
          \State \textbf{return} Match($captures'$)
        \EndCase
      \EndSwitch
    \EndCase
    \Case{{\tt jmp L}}
      \State \textbf{return} Backtrack($captures$, {\tt L}, $sp$)
    \EndCase
    \Case{{\tt match}}
      \State \textbf{return} Match($captures$)
    \EndCase
    \Case{{\tt skip} $n$}
      \State \textbf{return} Backtrack($captures$, $ip + 1$, $sp + n$)
    \EndCase
    \Case{{\tt save} $s$}
      \State $captures[s] \gets sp$
      \State \textbf{return} Backtrack($captures$, $ip + 1$, $sp$)
    \EndCase
    \Case{{\tt char} $\alpha$}
      \If{$sp < input.len() \wedge input[sp] \in \alpha$}
        \State \textbf{return} Backtrack($captures$, $ip + 1$, $sp + 1$)
      \Else
        \State \textbf{return} NoMatch
      \EndIf
    \EndCase
    \Case{{\tt scan-end} $l$}
      \If{$literal\_start \gets scan\_to\_start(l, input[sp..])$}
        \State \textbf{return}
            Backtrack($captures$, $ip + 1$, $literal\_start + len(l)$)
      \Else
        \State \textbf{return} NoMatch
      \EndIf
    \EndCase
    \Case{{\tt scan-begin} $l$}
      \If{$literal\_start \gets scan\_to\_start(l, input[sp..])$}
        \State \textbf{return}
                  Backtrack($captures$, $ip + 1$, $literal\_start$)
      \Else
        \State \textbf{return} NoMatch
      \EndIf
    \EndCase
    \Case{{\tt goto-end}}
      \State \textbf{return} Backtrack($captures$, $ip + 1$, $input.len()$)
    \EndCase
  \EndSwitch
\EndProcedure
\end{algorithmic}
\end{algorithm}

\begin{algorithm}
\caption{Bounded Backtracker}
\label{algo:loopboundedback}

\begin{algorithmic}
\State Let $input$ be the input string.
\State Let $insts$ be the instructions.
\Procedure{Backtrack}{$inst\_p$, $string\_p$}
  \State $seen \gets \{\}$
  \State $resume \gets [State(inst\_p, string\_p)]$
  \State $captures \gets [NONE, ...]$
  \While{$r \gets resume.pop()$}
    \Switch{$r$}
      \Case{$State(ip, sp)$}
        \If{$\langle ip, sp \rangle \in seen$}
          \State \textbf{break}
        \EndIf
        \State $seen \gets seen \cup \{\langle ip, sp \rangle\}$
        \State \verb'LOOP:'
        \While{$sp < input.len()$}
          \Switch{$insts[ip]$}
            \Case{{\tt split L1 L2}}
              \State $resume.push(State(L2, sp))$
              \State $ip \gets L1$
              \State \textbf{break}
            \EndCase
            \Case{{\tt match}}
              \State \textbf{return} true
            \EndCase
            \Case{{\tt skip} $n$}
              \State $sp \gets sp + n$
              \State $ip \gets ip + 1$
              \State \textbf{break}
            \EndCase
            \Case{{\tt save} $s$}
              \State $resume.push(Restore(s, captures[s]))$
              \State $captures[s] \gets sp$
              \State $ip \gets ip + 1$
              \State \textbf{break}
            \EndCase
            \Case{{\tt char} $\alpha$}
              \If{$input[sp] \in \alpha$}
                \State $ip \gets ip + 1$
                \State \textbf{break}
              \Else
                \State \textbf{break} \verb'LOOP'
              \EndIf
            \EndCase
            \Comment{Some cases omitted.}
          \EndSwitch
        \EndWhile
      \EndCase
      \Case{$Restore(slot, val)$}
        \State $captures[slot] = val$
        \State \textbf{break}
      \EndCase
    \EndSwitch
  \EndWhile
  \State \textbf{return} false
\EndProcedure
\end{algorithmic}
\end{algorithm}

\section{Skip PikeVM}

While it is usually not as fast as the bounded backtracker,
a modified version of Rob Pike's lockstep VM (\cite{Pike1987})
can also be used to execute skip regex. The skip Pike VM cannot
implement greedy vs lazy match differentiation, so it can't be
used in situations where differentiation is required.

The core idea of a Pike VM is to maintain just two sets of threads:
one situated at the current character and one at the next character.
In a standard NFA simulation, no thread will ever spawn with
a string pointer in any location besides these two, and
neither set of pending threads can ever grow larger than
the number of instructions, as duplicate threads are ignored.
Such a scheme requires a representation of a set of threads
between $0$ and $|regex|$ in size that supports fast containment
checking, fast iteration, and fast clearing. A sparse set using
uninitialized memory makes all of these operations constant except
for iteration which is linear (\cite{Briggs1993}), making it
a good choice for the representation of a thread set. Using
a set to hold active threads at a given index
effectively limits the number of steps which can occur
at any character of input to be $O(|regex|)$, guaranteeing that
the runtime does not exceed $O(|regex||input|)$. Typically the regex
is quite small compared with the input, so NFA simulation runtime is best
thought of as linear in the size of the input.

To support skip regex, the Pike VM must be able to deal
with threads in more than two locations, which requires expanding
the queue of runnable threads beyond just a pair of thread sets.
It is tempting to place all threads in a priority queue where
priority is determined by location in the input,
but this approach would cause the running time to become
$O(|regex||input|lg(|regex||input|))$, as retrieving the highest
priority thread would become logarithmic in the number of
of potential runnable threads rather than being constant.

To retain the algorithmic properties of the set of runnable threads,
we instead use a fixed size ring buffer of sparse sets, with
each set containing threads $n$ characters forward in the input,
where $n$ is the distance from the current sparse set. The fixed
size of the ring buffer means
that skips further ahead in the input than the length of the ring
buffer cannot be executed in one step, but we observe that most
skips are smaller than $60$, the ring buffer size that we choose
through benchmarking. It is more common for scan instructions to
require being repeatedly deferred to the furthest forward sparse set,
but we still observe a speedup in such cases.

Figure \ref{fig:pikeexecchar} contains an example that walks though
the representation of threads as the Pike VM steps through a simple
program that decides the regex \verb'/a/'. Here threads are represented as
a 2-tuple of an instruction pointer and an array of capture slots.
There is no need to store a string pointer because the thread set
which contains a thread implies its offset into the string.
Instructions like \verb'save' do not advance the string pointer,
so the first step of the Pike VM pops $t_0$, executes a save instruction
by saving the current string pointer to the first capture slot, then
pushes a new thread, $t_1$, to the current thread set. The second step
does advance the string pointer, so when $t_1$ is transformed into $t_2$
it is pushed to the thread set at index 1 instead of index 0.

Figure \ref{fig:pikeexecskip} contains an example that walks through
the execution of a simple program containing the \verb'skip' instruction,
which demonstrates the ability to jump forward in the input by pushing
a new thread to one of the thread sets further ahead in the ring buffer.
In such a case the skip Pike VM will notice that intermediate thread
sets are empty, and quickly skip over them.

\begin{figure}
\caption{Executing a {\tt char} Instruction in the skip Pike VM}
\label{fig:pikeexecchar}

\centering

\begin{tabular}{ l l }
\textbf{Instructions} &
  \begin{tabular}{| r | l |}
  \hline
  Index & Instruction \\ \hline
  0 & \verb'save 0' \\ \hline
  1 & \verb'char a' \\ \hline
  2 & \verb'save 1' \\ \hline
  3 & \verb'match' \\ \hline
  \end{tabular} \\ \\
\textbf{Step 0, Thread Set Index = 0} &
  \begin{tabular}{| c | c | c | c |}
  \hline
  $t_0 = \langle 0, [NONE, NONE] \rangle$ & & & \\ \hline
  \end{tabular} \\ \\
\textbf{Step 1, Thread Set Index = 0} &
  \begin{tabular}{| c | c | c | c |}
  \hline
  $t_1 = \langle 1, [0, NONE] \rangle$ & & & \\ \hline
  \end{tabular} \\ \\

\textbf{Step 2, Thread Set Index = 0} &
  \begin{tabular}{| c | c | c | c |}
  \hline
  & $t_2 = \langle 2, [0, NONE] \rangle$ & & \\ \hline
  \end{tabular} \\ \\

\textbf{End of Thread Set 0} & \\ \\

\textbf{Step 3, Thread Set Index = 1} &
  \begin{tabular}{| c | c | c | c |}
  \hline
  & $t_3 = \langle 3, [0, 1] \rangle$ & & \\ \hline
  \end{tabular} \\ \\

\textbf{Match} & \\

\end{tabular}
\end{figure}

\begin{figure}
\caption{Executing a {\tt skip} Instruction in the skip Pike VM}
\label{fig:pikeexecskip}

\centering

\begin{tabular}{ l l }
\textbf{Instructions} &
  \begin{tabular}{| r | l |}
  \hline
  Index & Instruction \\ \hline
  0 & \verb'save 0' \\ \hline
  1 & \verb'skip 3' \\ \hline
  2 & \verb'save 1' \\ \hline
  3 & \verb'match' \\ \hline
  \end{tabular} \\ \\
\textbf{Step 0, Thread Set Index = 0} &
  \begin{tabular}{| c | c | c | c |}
  \hline
  $t_0 = \langle 0, [NONE, NONE] \rangle$ & & & \\ \hline
  \end{tabular} \\ \\
\textbf{Step 1, Thread Set Index = 0} &
  \begin{tabular}{| c | c | c | c |}
  \hline
  $t_1 = \langle 1, [0, NONE] \rangle$ & & & \\ \hline
  \end{tabular} \\ \\

\textbf{Step 2, Thread Set Index = 0} &
  \begin{tabular}{| c | c | c | c |}
  \hline
  & & & $t_2 = \langle 2, [0, NONE] \rangle$ \\ \hline
  \end{tabular} \\ \\

\textbf{End of Thread Set 0} & \\ \\

\textbf{End of Thread Set 1} & \\ \\

\textbf{End of Thread Set 2} & \\ \\

\textbf{Step 3, Thread Set Index = 3} &
  \begin{tabular}{| c | c | c | c |}
  \hline
  & & & $t_2 = \langle 2, [0, 3] \rangle$ \\ \hline
  \end{tabular} \\ \\

\textbf{Match} & \\

\end{tabular}
\end{figure}

\section{Choosing a Backend}

For skip expressions, the choice between backends is generally
pretty easy. The bounded backtracker is almost always faster
than the Pike VM, and the Pike VM will not predictably
differentiate different matches. The question is really,
``when would you ever want to use the Pike VM?'' rather than
``which backend should you use?''. If an expression has
no branches which might impact the capture groups of
the expression\footnote{For example, all branches might appear
within the only capture group.} and both the input and
the regular expression are large enough that the cost of
zeroing the backtracker's bitset becomes too high, it
makes sense to use the Pike VM. Determining these cases
are tricky, and we don't currently attempt to do
so automatically. Most users will want to just select
the bounded backtracker.

In the context of a regular expression engine which does not
mind a potentially exponential runtime, such as pcre2, there
would never be any reason to make use of the Pike VM. In such
a case, the backtracker would not have a bitset to prevent
exponential blowups.
