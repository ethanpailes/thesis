\chapter{Future Work}
\label{chapter:futurework}

There is significant opportunity for skipping over more complicated
expressions than just literals. The simplest case is skipping over
alternatives where each branch is of the same length. A more interesting
problem is adding AST rewrite passes which factor common prefixes and
suffixes out of alternatives in order to increase the length of skips.
For example \verb'/a(?:bbc|bbyx)/' might could be rewritten as
\verb'/abb(?:c|yx)/'.

Both the skip regex optimizations presented in this thesis and
Chivers' notion of previews (\cite{Chivers2016}) represent
extensions to the VM approach to NFA simulation. The approaches
are compatible, so it would be interesting to see how an engine
which makes use of both sets of optimizations performs. Beyond
just being compatible with skip regex, regex previews have
synergistic potential with them. In particular, previews make
it possible to quickly scan forward to a point which might begin
any regex, which ought to allow an expansion of the cases where
the scan optimizations presented in this thesis are applicable.

It would be nice to add support for unicode and case folding to
the skip regex engine, as this would allow skip regex to be used
as a truly drop-in replacement for the Rust regex crate\footnote{
When used in conjunction with a DFA to filter out non-matching
input}.

The original motivation for investigating parsing a regex without
deciding it was the need to perform a partial parse for the PADS
language. Now that the problem of partial parsing has been worked
out for regular expressions, a natural next step is to see how
the work extends to a language like PADS. Of particular interest
is the ability to decide when two PADS grammars have an empty
intersection. PADS grammars tend to have a bit more structure
than regex, so it seems like the $e*$ optimization has more
potential than the $.*$ scan optimization. The ability to check
if two pads grammars have an empty intersection is a key piece
of analysis for enabling such an optimization. Another angle
to consider is the fact that PADS grammars are not regular,
which opens up the potential for optimizations. To scan
forward over a parenthesized s-expression something like
algorithm \ref{algo:sexpscan} seems to be the
optimal solution. This sort of counting scan can't come up
in the context of regular expressions due to their regularity,
but it must be considered for PADS.

\begin{algorithm}
\caption{S-Exp Scan} \label{algo:sexpscan}
\begin{algorithmic}
\Procedure{SExpScan}{$i$}
  \State $num\_parens \gets 0$
  \State $in\_quote \gets false$
  \While{true}
    \If{$input[i] = '('$}
      \If{$\neg in\_quote$}
        \State $num\_parens \gets num\_parens + 1$
      \EndIf
    \ElsIf{$input[i] = ')'$}
      \If{$\neg in\_quote$}
        \State $num\_parens \gets num\_parens - 1$
      \EndIf
    \ElsIf{$input[i] = '"'$}
      \State $in\_quote \gets \neg in\_quote$
    \EndIf
    \State $i \gets i + 1$
    \If{$num\_parens = 0$}
      \State \textbf{break}
    \EndIf
  \EndWhile
  \State \textbf{return} i
\EndProcedure
\end{algorithmic}
\end{algorithm}

% TODO: to facilitate optimization-aware programming I should make it
%       easier to understand when a particular optimiztion is applied.
