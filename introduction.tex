\chapter{Introduction}
\label{chapter:introduction}

Pattern matching is a powerful tool for concisely and declarativly
describing expected data. In the context of programming languages,
pattern matching enables programmers to directly encode the shape
of data they are interested in. The applications for textual pattern
matching are even more widespread. Textual patterns are used for
editing, \textit{ad-hoc} parsing, url route dispatch, and a variety of
bioinformatics processing tasks. This demand for textual pattern
matching has led to a rich and evolved body of work on the subject.

\section{The Problem: Partial Parsing}

At their core, regex know how to answer two questions: ``Is this
input in the language?'' (decision) and ``What are the capture
groups?'' (parsing). Existing implementations can answer the former
without the latter, but never answer the latter without checking
the former. In cases where regular expressions are being used to
parse text that is already known to match, there is extra work being done
to re-validate the input along with parsing. If this work can be
elided, there is a great deal of opportunity for optimization.
This thesis aims to explore how possible it is to separate regular
expression parsing from the regular expression decision problem,
and to show that focusing entirely on parsing while paying decision
no heed can yield speedups. We investigate the benefits of this
separation when both problems must be solved as well as when
there is no need for decision.

Stated more formally, the problem we set out to solve is to find
the most efficient way to compute $captures(e, w)$ where $e$ is
a regular expression, $w \in L(e)$ where $L$ is a function from
a regular expression to its language, and $captures$ is a function
which returns the capture groups of a regular expression after
a parse.

To solve this problem, we present skip regex, a new regex library,
or engine, which leverages the assumption that the input is in the language
of the regex being parsed. Skip regex extends the regular expression
backend with the ability to skip forward to an arbitrary point in the
input.

\subsection{The Partial Parsing Assumption}

We define the partial parsing assumption to mean that the
input being parsed is in the language being parsed.

\section{Motivation}

The original motivation for this work comes not from the realm of
regular expressions, but from work on PADS, a language for ad-hoc
data description (\cite{Fisher2005}), and Forest, a language for
principled programming with ad-hoc file oriented databases
(\cite{Fisher2011}). The need for a partial parse first
came up in the context of Forest where queries sometimes depend on
a small subset of a very large corpus of data stored in the filesystem.
Forest uses PADS for parsing the contents of specific files, so the
natural next step was to think about how partial parsing might
be used in the context of ad-hoc data representations.
A partial parse asks for a certain subset of a full parse tree,
and allows any intervening data to be thrown out. It is impossible
to perform a partial parse without making the partial parsing assumption.
The partial parsing assumption seems like an odd prerequisite for
a parsing problem, as parsing is in large part about deciding
languages. The whole purpose of a partial parse
is to aid performance by only looking at the parts of the input which
are absolutely necessary to understand the subset of the parse tree
that we are really interested in. If we avoid looking at any part of
the input, that portion of the input might contain syntax errors.
Then if we want to only examine a subset of the input we must assume
that the input is in the language being parsed. This should make it
clear that the partial parsing assumption is necessary to make
any progress in partial parsing.

Compared with PADS, regular expressions have a small and
constrained semantics which makes rapid iteration on new
ideas easier. Regular expressions already have a very natural
starting place for working on partial parsing built into their
semantics in the form of capture groups. In fact all regex parses
can be thought of as partial parses because capture groups ask for
a subset of the matching input. The idea that the user wants
to know about the things inside capture groups but not outside of
them aligns perfectly with the ideas we were developing about
partial parsing. Any complete consideration of partial
parsing in PADS will have to deal with partial parsing in regular
expressions anyway, as regular expressions are important primitives in
a PADS grammar. All of these points pointed to regular expressions as a good
starting point for research in partial parsing.

There is also ample reason to examine partial parsing purely
within the context of regular expressions. Situations where the input
will always match, such as parsing csv files
or processing the output of a command line tool in a unix pipeline
are the most obvious application. More interestingly, the
relative speed of DFA vs NFA execution makes regex partial parsing
useful for regex parsing in general. DFAs can decide regular expressions,
but not parse them or support perl-style back references. Despite
these limitations, they are still used extensively by tuned regex
implementations because they are much faster than direct NFA
simulation (\cite{CoxRE2}, \cite{GallantRegex}, \cite{GoLang}).
In fact, this difference is so great that in situations requiring
an NFA some engines will first use a DFA to find the subset of the input
which matches and then run an NFA on the matching (and hopefully
smaller) input. Thus, the work in this thesis is applicable
to regular expression parsing regardless of whether or not the user
can actually guarantee that the input matches the expression.

A final motivation to consider is the possibility for use
in parsing trusted text. Programs often make a distinction between
trusted and untrusted data, and then proceed to perform operations
on trusted data that would be wildly unsafe on untrusted data.
The most obvious example is in-memory data structures such as C structs, but
it is also quite common with more modern text types such as
a Python 3 \verb'string' or a Rust \verb'&str'. In Rust, programmers
are allowed to perform several operations on a \verb'&str' which
would be unsafe if it were not \verb'UTF-8', so when a \verb'&str'
is first constructed the programmer must either ask Rust to validate
the data or undertake a proof obligation that the data is good.\footnote{
Via Rust's {\tt unsafe} system.} For systems that leave
their data in textual form while in memory\footnote{
Whether to save memory, avoid the cost of a full parse, remain
flexible in the face of a possibly changing textual format, or
just continue to work with poor legacy engineering decisions.}
the ability to impose this common taxonomy between trusted and
untrusted data could prove useful.

\section{Claims}

\subsection{Skip Regex are Fast}

Skip regex can be used on their own or
in concert with a DFA regex engine to get speedups for
regex partial parsing or full regex parsing respectively.
Skip regex are an optimization, and optimizations are not worthwhile if
they don't provide an actual performance win. To demonstrate
the performance characteristics
of skip regex, we provide micro benchmarks in section
\ref{section:microbenchmarks} to showcase the specific
capabilities of skip regex, case studies in section
\ref{section:logparsingcase} to demonstrate their
usefulness and performance in more realistic scenarios, and a
discussion of their time complexity.

\subsection{Skip Regex are Useful}

Skip regex are only interesting if they are useful to programmers.
The easiest way that the skip regex engine could be useful to programmers
would be if its optimizations were applicable to the sorts of regex they
are already writing. To check for the applicability of skip optimizations
on existing regex we scraped \verb'crates.io',
the Rust library repository, and checked
to see which optimizations
we could apply to each regex we found (section \ref{section:applicability}).
Skip regex open up new programming
possibilities by allowing users to make a distinction between trusted
and untrusted textual data. New programming approaches cannot be 
examined by looking at existing regular expressions, so we provide
case studies to examine the possibilities of this new approach.

Regular expressions are a well established tool, so any change in
semantics would violate the principle of least surprise.
If the semantics of skip regex remain backwards compatible with
existing implementations, users will find them easy to pick up.
Backwards compatibility also leaves the door open for providing
a skip regex based drop-in replacement for existing implementations.
The biggest barrier to the drop-in replacement use case is the
assumption that the input matches the expression. Fortunately,
this issue can be overcome by using a DFA to filter out non-matching
input. To demonstrate backwards compatibility we tested our implementation
and argue informally that the optimizations we perform are
semantics-preserving.

% TODO: explain what the different chapters of the thesis are going to
%       talk about.
