\chapter{Conclusion}
\label{chapter:conclusion}

The primary objectives of this thesis were the production of a
fast and useful implementation of regular expression parsing,
and laying theoretical groundwork for future work in partial
parsing.

We examined the performance profile of skip regex in detail
using microbenchmarks, and provided application level performance
numbers through case studies. Both types of benchmarks suggest
that skip regex can produce a speedup over existing
implementations. The case studies we present provide examples
of how skip regex might be applicable, but to further
the claim of applicability we show that over 80\% of regex
found on \verb'crates.io' can benefit from some sort of
skip optimization.

In addition to being a suitable replacement for standard
regex with the help of a DFA filter, skip regex open up
a new approach to programming. Programmers have long made
the distinction between trusted and untrusted binary data.
Skip regex provide a systemic approach to handling trusted
textual data. The usefulness of this distinction is hard to
evaluate without giving programmers time to use the new tool,
but the existence of the analogous distinction with respect to
binary data is promising.

This thesis uncovered a number of insights that ought to be
useful for examining partial parsing in the context of PADS.
We have shown that while substring search is in the same complexity
class as regular expression parsing, the difference in constant
factors is so huge that looking for opportunities to scan forward
to some literal is very important. The work on analyzing the
ambiguity at branch points and checking for regex intersection
ought to be useful when attempting to analyze PADS grammars
for optimization.

Skip regex are fast, applicable to real applications, open
up a new approach to programming with trusted text, and lay
the groundwork for further work in partial parsing.
