\chapter{Regular Expressions and an Abstract VM for Their Execution}
\label{chapter:abstractvm}

The work presented in this thesis focuses on improvements to the
speed with which NFAs can be simulated through the use of a
specialized virtual machine, a common approach to regular
expression implementation (\cite{CoxVirtualMachineApproach}).
This chapter introduces the regular expression grammar we will
work with, and the abstract virtual machine that we will target for
execution. Implementation of this machine will be covered in chapter
\ref{chapter:implementation}.

\section{Regular Expressions}
\label{section:regexdef}

Before going into the semantics of the virtual machine, we
define the regular expression grammar we will focus on.
Keeping the syntax and semantics of regular expressions in
mind should help motivate the structure of the VM we present
in section \ref{section:vm}.

A regular expression can decide whether or not a given string
matches and which substrings of the input correspond to particular
subsections of the regular expression, called capture groups.
Here $\Sigma$ represents the alphabet from which characters in
the input string are drawn.

\begin{grammar}
  <e> ::= $e e$
  \alt $e \rvert e$
  \alt $(e)$
  \alt $\alpha$ {\tt where} $\alpha \subset \Sigma$
  \alt $e*$
  \alt $e+$
  \alt $e*?$
  \alt $e+?$
  \alt $.$
\end{grammar}

$e e$ represents concatenation of the two expressions,
$e|e$ represents choice between the two expressions, $(e)$ indicates a
capture group around $e$, $e*$ represents the Kleene closure of $e$,
and $a$ indicates a regular expression which matches a character set.
These forms alone are enough to talk about regular expressions, as
the remaining forms can all be implemented in terms of this core set.
A simpler formulation is usually preferred for theoretical work because
it provides a smaller surface area that proofs must work with. 

Of the remaining forms, $e+$ represents $e$ repeated 1 or more times,
$e*?$, called lazy Kleene star, is a version of $e*$ which always prefers
the shortest match rather than the longest, $e+?$ is a version of
$e+$ which prefers the shortest match, and $.$ is a convenient
way to say $a$ where $a = \Sigma$. These more complex forms are
included despite the theoretical preference for simplicity in
order to facilitate discussion of optimizations.

Some of the more complex forms found in the source language of real
world regular expressions can be implemented more efficiently 
when done so directly
than they would otherwise be if first translated to the
simpler form. For example, $e*$ requires an extra branch when compared
with $e+$ to account for the possibility of zero repetitions.
In order to fully explain the details of each of the
optimizations presented below, we therefore retain some of the more
interesting higher level forms. The implementation of greedy vs lazy
repetition ($e*$ vs $e*?$) differs
slightly, so we also include the extra repetition forms in order to
address those differences below.

Typical abstract regular expression syntax lacks a way to talk about
precedence separate from capture groups. For the sake of simplicity,
we have not added such a form to the above grammar, but it remains
useful to be able to write example regular expressions with explicit
precedence. In such cases we will use $(?:e)$ to indicate a grouping
of $e$ without a corresponding capture. This is the same concrete syntax
provided by the Rust regex crate.

\subsection{Literals}

Literals are fixed strings of characters of length one or more.
Stated differently, $l$ is called a literal if $l = a+$ for $a \in \Sigma$.

\section{The Regular Expression VM}
\label{section:vm}

One approach to regular expression matching is to compile regular expressions
to a set of bytecode instructions for a specially designed VM. There
are several different potential VM implementations, each with its own
set of trade offs, but each of these back-ends share the same bytecode
and compiler front end. In order to understand the compiler, it
is important to first develop an understanding of the abstractions
that the target machine provides.

Regex VM instructions are executed in the context of a VM thread.
A VM thread consists of an instruction pointer, which indicates
the next VM instruction to be executed, an input cursor which
indicates the current position in the input to be matched, and
a finite number of capture slots. Capture slots are used to
track the start and end of capture groups with one slot for each
end of the capture. After the regex execution, the capture groups
are reconstructed from these slots. The \verb'save' instruction is
responsible for placing the appropriate breadcrumbs in the slots.
Threads may spawn multiple child threads at once which are logically
non-deterministically executed. This first class non-determinism 
makes it possible to easily encode the natural non-determinism of
a regular expression. The main project of regex VM back-ends is
efficiently managing the non-determinism.

VM instructions can be thought of as belonging to three different categories:
control flow, tests, and bookkeeping. Control flow instructions deal with
spawning, killing, and modifying threads. Tests examine the input to determine
whether the current thread can survive. Bookkeeping instructions track metadata,
in particular information about where capture groups begin and end.
Many of the instructions come from standard regex simulation
techniques, but several have been added to support the optimizations
presented in this thesis. Descriptions of the standard instructions 
can be found in table \ref{table:insts}, while descriptions of the
extensions can be found in \ref{table:skipinsts}.

\subsection{Code Blocks and Addresses}

Regex VM instructions are laid out in a linear fashion, just like
machine code for real processors. A VM program consists of a list of
VM bytecodes, each of which is addressed by its offset. During
execution, an instruction pointer indicates the current instruction.
After an instruction has been executed, the instruction pointer
is incremented unless the definition of the instruction (such as
the \verb'jmp' instruction) says otherwise. Labels can be used
in place of fixed addresses to provide convenient names for
use in branching code. A label is equivalent to the address
of the next instruction which appears.

\begin{table}[ht]
\centering
\caption{Core VM Instructions}
\label{table:insts}

This table defines the VM instructions used by standard regex engines.
Our extensions to the instruction set can be found in table
\ref{table:skipinsts}.
\begin{tabular}{| l | c | p{9cm} |} \hline
Instruction & Inst Type(s) & Description \\ \hline
{\tt split $l_1$ $l_2$} & Control Flow &
    \verb'split' tells the VM to spawn new threads, $t_1$ at $l_1$ 
    and $t_2$ at $l_2$. The parent thread dies.
    If $t_1$ or any of its children match and
    $t_2$ or any of its children match, the VM must return
    the match from $t_1$. Thus the code:
        \begin{verbatim}

split L1 L2
L1: save 2
    char {x|y}
    save 3
L2: save 4
    char {y|z}
    save 5
match
        \end{verbatim}
    executed on \verb'y' would result in the first capture group
    (specified by slots 2 and 3) being populated with a \verb'y'
    even though both branches would result in a match. On the
    other hand, if the labels in the split instruction were
    flipped, the second capture slot (specified by slots 4 and 5)
    would be populated with \verb'y' instead.

    Actual implementations often choose to mutate the parent thread in
    place and only spawn a single new thread.
    \\ \hline
{\tt jmp $l_1$} & Control Flow &
    \verb'jmp' tells the VM to set the instruction pointer
    to $l_1$. Note that the input cursor does not move. \\ \hline
{\tt match} & Control Flow &
    \verb'match' indicates a match. The VM may stop. The behavior of match
    is slightly complicated by the presence of an end anchoring
    (\verb'$'). If the regex is anchored at the end, a match instruction
    only counts if the string pointer indicates the end of the input.
    \\ \hline
{\tt save $s$} & Bookkeeping &
    \verb'save' tells the VM to save the current string
    offset into capture slot $s$. \\ \hline
{\tt char $\alpha$} & Test &
    \verb'char' tells the VM to kill the current thread unless
    the current char is in $\alpha$ (where $\alpha \subset \Sigma$).
    If the current char is in $\alpha$, the VM must advance the input cursor
    for the current thread. \\ \hline
\end{tabular}
\end{table}



\begin{table}[ht]
\centering

\begin{tabular}{ | l | c | p{8cm} | } \hline
Instruction & Inst Type(s) & Description \\ \hline
{\tt skip $n$} & Control Flow &
    \verb'skip' tells the VM to advance the input cursor by $n$.
    Skip is an extension
    instruction provided specifically to facilitate the optimizations
    presented here. \\ \hline
{\tt scan-end $l$} & Test \& Control Flow &
    This test instruction tells the VM to scan forward in the input to find
    the literal $l$. If $l$ is not found, the VM must kill the current thread,
    otherwise it must place the input cursor
    right after the end of $l$ in the input. For example executing
        \begin{verbatim}
        scan-end ``foo''
        \end{verbatim}
    on \verb'xxxxfooyy' would place the string pointer at the first \verb'y'
    char. \verb'scan-end' is an extension instruction.

    While a test instruction, the \verb'scan-end' instruction can
    significantly advance the input cursor, so it can also be classified
    as a control flow instruction. \\ \hline
{\tt scan-begin$l$} & Test \& Control Flow &
    This instruction behaves just like \verb'scan-end' except that the
    VM must place the input cursor at the beginning of the literal on
    a successful match. \verb'scan-begin' is an extension instruction.
    \\ \hline
{\tt goto-end $l$} & Control Flow &
    This instruction tells the VM to just jump to the end of the input.
    It is included to facilitate optimizing trailing $.*$ expressions.
    \\ \hline
\end{tabular}

\caption{Skip VM Instructions}
\label{table:skipinsts}
\end{table}

\section{Nondeterministic Finite Automata}
\label{section:nfadef}

While most of the work in this thesis will focus on the representation
of an NFA as a VM program, it is still sometimes useful to talk about
a more traditional representation. We offer
a brief definition of an NFA here.

An NFA $n$ is defined as a 5-tuple of the form
$\langle Q, \Sigma, \Delta, q_0, F \rangle$ where $Q$ is the
set of states, $\Sigma$ is the alphabet, $\Delta$ is the transition
function, $q_0$ is the initial state, and $F$ is a set of accepting
states. We define the $\Delta$ function as a multiset of elements
with the form $\langle q_f, c \rangle \rightarrow q_t$ where
$q_f$ and $q_t$ are states in $Q$ and $c$ is a character in $\Sigma$.

